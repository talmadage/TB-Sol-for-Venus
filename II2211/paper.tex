\documentclass[10pt,twocolumn]{article}

% Pacchetti per font e formattazione
\usepackage{times}        % Times New Roman
\usepackage[utf8]{inputenc}
\usepackage[T1]{fontenc}
\usepackage{geometry}
\geometry{a4paper, margin=2cm}
\usepackage{titlesec}
\usepackage{multicol}
\usepackage{cite}

% Titoli sezioni (formato)
\titleformat{\section}{\bfseries\fontsize{12}{14}\selectfont}{\thesection.}{0.5em}{}
\titleformat{\subsection}{\bfseries\fontsize{11}{13}\selectfont}{\thesubsection.}{0.5em}{}

% Dati frontespizio
\title{\textbf{TB for Venus RISC-V}\\[0.5em]}

\author{Chinmay Purandare\\
Lorenzo Parata\\
Chongnan Wang\\
Xiaotian Jiang \\
Xinxin Qin \\
Department of Electronics\\
KTH Royal Institute of Technology\\
Stockholm, Sweden\\
Emails: chinmayp@kth.se, parata@ug.kth.se, chongnan@kth.se, xjia@kth.se, xinxinq@kth.se}

\date{}

\begin{document}

\maketitle

\section*{Abstract}
This project presents the development of a testbench framework for the Venus RISC-V initiative at KTH Royal Institute of Technology. The objective is to validate two FPGA-based processors—a parallel RISC-V core with a bit-serial cache interface, and a bit-serial “CISC-V” variant—under realistic hardware and verification constraints. The system separates the processor and the testbench onto two FPGAs, simulating the communication bottlenecks expected in future SoI-based high-temperature implementations for Venus surface missions. Following an agile project model, the team designed, implemented, and verified the hardware through incremental sprints. The final prototype successfully demonstrates a reproducible hardware test environment capable of validating functional correctness, communication reliability, and performance feasibility for simplified RISC-V cores targeting extreme environments.

\section*{Keywords}
RISC-V CISC FPGA Venus

% Sezioni principali
\section{Introduction}
The exploration of Venus surface missions requires electronic systems that can survive extreme conditions of up to 480°C, high pressure, and a corrosive atmosphere. Conventional silicon technology fails at such temperatures, motivating the use of Silicon-on-Insulator (SoI) and Silicon Carbide (SiC) technologies. However, validating designs for these technologies in realistic environments is expensive and slow. Therefore, FPGA-based emulation and verification provide a cost-effective platform for early design validation before fabrication.\\
This project aims to build a hardware testbench that can evaluate bit-serial RISC-V processors intended for high-temperature deployment. Two cores are considered: a parallel RISC-V model and a bit-serial “CISC-V” core implemented using the ProGram tool. The project focuses on their functional verification, performance characterization, and system-level integration. The design methodology follows an agile approach, structured in short iterations (sprints) emphasizing early testing, modular development, and continuous risk assessment.\\
Ethically, the project follows academic research standards: all external intellectual property (e.g., RTL, tools, frameworks) is used under proper open-source licenses. From a sustainability perspective, the project supports low-power, resource-efficient computing, a key enabler for future energy-constrained space missions.

\section{Related Work}
This section reviews research relevant to developing a testbench for bit-serial RISC-V processors targeting Venus surface applications. We examine RISC-V architectures, high-temperature electronics, FPGA implementation methods, and verification frameworks. These foundations inform the design of a testbench capable of validating processor functionality under extreme environmental constraints.

\subsection{RISC-V}
Researchers from TU graz developed FazyRV \cite{kissich2024fazyrv}, a minimal-area open-source RV32I RISC-V core targeting IoT and low-workload applications, addressing the problem that 32-bit RISC-V processor cores reach a boundary on their minimal size. Their goal was to minimize area demand while fulfilling performance requirements and close the gap between prevalent 32-bit and 1-bit-serial RISC-V cores.\\
Qian Wei and his colleagues created a comprehensive survey of RISC-V instruction set architecture extensions because while RISC-V is popular for embedded processors \cite{cui2023riscv}, there is still a gap between RISC-V's capabilities and the requirements of various emerging computing scenarios like artificial intelligence and cloud computing.\\
Gautschi et al. conducted a comprehensive comparison of ultra-low-power RISC-V cores for Internet-of-Things applications \cite{gautschi2017slow}, analyzing the trade-offs between performance and energy efficiency in resource-constrained environments. Their work provides valuable insights into processor design considerations for applications with tight area and power constraints, which directly supports the rationale for bit-serial processor implementations in challenging deployment scenarios such as the 12-pad limitation required for Venus surface operations.

\subsection{Processors for Venus}
Current research on high-temperature electronics for Venus has shown that SiC ICs can sustain operation for over a year at 500 °C and for months in simulated Venus conditions, supporting sensors and simple microprocessors. This demonstrates the feasibility of long-duration surface missions and motivates concepts like LLISSE targeting low-power seismic monitoring \cite{kim2022sic}. \\
At the processor level, studies of SiC-based computing infrastructures reveal that current prototypes achieve significantly lower throughput than space-proven silicon systems, yet provide guidelines at the microarchitecture and ISA levels for developing processors able to operate under Venus's extreme environment \cite{kremic2021temperature}.\\
Pradhan and colleagues provided a comprehensive review of materials for high-temperature digital electronics \cite{pradhan2024materials}, highlighting the challenges and solutions for electronic systems operating at temperatures as high as 500°C and beyond. Their work emphasizes the critical importance of developing new material solutions beyond conventional silicon-based devices for applications including space exploration, with specific attention to the extreme conditions encountered in Venus surface missions.

\subsection{FPGA}
One verification strategy is co-emulation, where the RTL design on the FPGA is run in parallel with a trusted software model, such as the Spike simulator \cite{moreno2023fpga}, on a host PC. This allows for high-speed verification by comparing the core's architectural state against the simulator in real-time. Furthermore, using FPGAs with RISC-V is advantageous as it allows for optimized hardware, where the FPGA is configured with only the peripherals required for a specific application \cite{efinix2023riscv}.

\subsection{FPGA-Based Verification Frameworks}

Kim \cite{kim2019fpga} demonstrated the importance of FPGA-based acceleration for RTL evaluation by introducing automated flows that generate cycle-accurate simulators directly from RTL. This approach reduces the engineering effort compared to earlier manual FPGA frameworks and enables both faster and more reliable pre-silicon verification. 

Building on this direction, Qin \textit{et al.} \cite{qin2025feriver} proposed FERIVer, an FPGA-assisted framework for verifying RISC-V processors. Their method exploits the heterogeneous architecture of SoC FPGAs, running an instruction set simulator on the processing system in parallel with the RTL core on the programmable logic. By cross-checking execution states, FERIVer achieves significant speedups over traditional software simulators while maintaining accuracy. 

Together, these works highlight how FPGA-based infrastructures provide an effective foundation for validating processor designs before costly SoC fabrication, a goal directly aligned with the testbench methodology developed in this project.


\section{Specification}
The project is constrained by both hardware and environmental limitations. The test setup consists of two FPGA boards, one implementing the processor core and the other acting as the testbench. Due to the 12-probe limitation of the physical test station, signal interfaces are highly constrained, motivating bit-serial communication schemes.\\
The design goals are:\\
\begin{itemize}
    \item Maintain compatibility with RISC-V base ISA (RV32I).
    \item Ensure modularity between processor and testbench components.
    \item Enable automated test execution and result logging.
    \item Achieve operational stability for extended FPGA runs to emulate continuous workload.
\end{itemize}
Hardware synthesis is performed using vendor FPGA tools and verified using ModelSim/QuestaSim. The ProGram compiler is used for the bit-serial CISC-V generation. Documentation, risk tracking, and test automation are maintained via version control and scripted pipelines.\\

\section{Architecture}
The overall architecture is divided into two major subsystems connected through a dedicated communication interface:
\begin{itemize}
\item Processor FPGA – hosts either the parallel RISC-V or the bit-serial CISC-V core.
\begin{itemize}
    \item The RISC-V core implements a bit-serial cache interface to mimic the narrow I/O expected under physical constraints.
    \item The CISC-V processor is generated from a high-level specification using the ProGram tool and synthesized into FPGA fabric.
    \item Both processors share a minimal peripheral set (instruction memory, serial link, debug UART).
\end{itemize}
\item Testbench FPGA – provides I/O emulation, input stimulus, and result logging.
\begin{itemize}
    \item Implements command control, data loading, and timing synchronization with the processor FPGA.
    \item Generates and monitors test patterns to verify the correctness of arithmetic, control flow, and memory operations.
    \item Logs execution traces for analysis and regression verification.
\end{itemize}
\end{itemize}
The communication between the two FPGAs uses a synchronous serial protocol, minimizing the number of required pads and simplifying PCB connectivity. Each subsystem is independently testable, allowing early functional verification before full integration. The design adopts incremental hardware-in-the-loop testing and leverages a modular top-level hierarchy for flexible reconfiguration.

\section{Experiments}
A structured experimental campaign was conducted to validate the system functionality. Each major milestone corresponded to a separate experiment phase:
\begin{itemize}
  \item \textbf{Core-A validation} : The parallel RISC-V processor was synthesized and tested with simple benchmark programs (e.g., arithmetic, loop, and branching tests) using a single FPGA. Functional correctness was verified via waveform comparison and on-board UART output.
  \item \textbf{Testbench bring-up} : The second FPGA was programmed with the testbench, initially tested in isolation through internal loopback and signal toggling to validate timing and I/O drivers.
  \item \textbf{Dual-FPGA integration} : The communication link between the two FPGAs was established, allowing the processor to execute test programs under the control of the testbench. Signal integrity, synchronization, and latency were evaluated through test logs and signal probing.
  \item \textbf{Bit-serial CISC-V verification} : The ProGram-generated core was synthesized, and functional tests were executed under the same testbench configuration. Comparative measurements against the parallel RISC-V implementation were made to observe performance and throughput differences.
  \item \textbf{Regression and stability testing} : Automated test scripts executed repeated workloads to ensure reproducibility and detect intermittent synchronization or logic issues.
\end{itemize}
All tests were performed iteratively under the agile process, where each sprint concluded with a functional demonstration and documentation update.

\section{Results}
The implemented testbench successfully validated both processor variants on FPGA hardware. The parallel RISC-V achieved stable operation and consistent test completion, while the bit-serial CISC-V demonstrated functional correctness albeit at reduced throughput, confirming the expected performance-area trade-off.\\
Integration over the dual-FPGA configuration achieved reliable end-to-end communication with reproducible data transfer and command execution. The testbench’s automated logging framework enabled efficient verification and provided traceable results across sprints.\\
Through repeated testing, the team confirmed the feasibility of conducting hardware-in-the-loop verification for simplified RISC-V processors under severe I/O constraints, demonstrating that the proposed testbench design is robust, flexible, and extensible to future SoI-based high-temperature platforms. The results also validate that the system can serve as a generic framework for testing future processor iterations or SoC components before fabrication.

\section{Conclusion}
This project developed a comprehensive FPGA-based testbench for validating bit-serial RISC-V processors aimed at future Venus surface applications. Through systematic design, sprint-based development, and risk-aware planning, the team successfully realized a modular dual-FPGA platform that enables early-stage verification of simplified processor architectures under constrained I/O and environmental assumptions.\\
The key outcomes include (1) functional verification of both parallel and bit-serial processors, (2) a reusable hardware testbench with automation support, and (3) a validated design methodology applicable to SoI and SiC environments. The project demonstrates the effectiveness of combining agile methods with hardware prototyping to manage complexity and mitigate technical risks in research-oriented processor development.\\
Future work will extend the framework toward temperature-aware modeling, SoI integration, and FPGA co-emulation with instruction-set simulators to achieve higher verification coverage and stronger correlation with final silicon implementations.

\bibliographystyle{ieeetr}
\bibliography{references}

\end{document}
